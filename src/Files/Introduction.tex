\section{Introduction}
\label{sec:introduction}
\subsection{Background}

Mobile robots are gaining popularity, particularly in non-industrial applications such as military, disaster management, and home applications. This growth is driven by advancements in technology and the motor vehicle industry. Like traditional vehicles, mobile robots can be classified based on their type of motion. Most mobile robots have rectilinear motion, similar to regular cars. However, other platforms have omnidirectional motion, allowing them to move in any direction at any point in time or holonomic motion allowing them to move without changing the orientation of the platform's body. This type of motion is known as holonomic, the robots can be controlled by degrees of freedom equal to the total degrees of freedom of the mobile robot. A good example of this type of motion is the mecanum wheel. Another example that inspired this project is the caster wheel. This project proposal aims to apply the concept of omnidirectional and holonomic motion in caster wheels to develop a mobile platform with potential applications in commercial settings such as industries, hospitals, and supermarkets. The goal of this project is to design and fabricate a mobile platform with omnidirectional and holonomic motion using caster wheels and to evaluate the control challenges associated with this type of motion. The results of this project could contribute to the advancement of omnidirectional and holonomic control of mobile platforms.

\subsection{Problem statement}
Mobile robots and platforms have found general applications in homes and other non-industrial applications. These vehicles are largely unmanned and operated remotely. This technology can be adopted for commercial use by improving on caster technology. Casters are used to move heavy and large objects on the warehouse or factory floor. This process is manual and the operator has to push around the caster physically. Casters require an initial push force to begin rolling. Furthermore, labourers get tired easily when pushing the heavy casters around especially due to difficulty in maintaining the correct swivelling. The need, therefore, arises to develop unmanned mobile platforms that use caster wheels which is able to achieve holonomic and omnidirectional motion.

\subsection{Objectives}
\label{sec:objectives}
\subsubsection{Main Objective}
To design and fabricate a prototype mobile platform capable of omnidirectional and holonomic motion.

\subsubsection{Specific Objectives}

\begin{enumerate}
    \item To design and build a mechanical chassis and body that will hold the wheels and carry the bulk of the load respectively.
    \item To design and build a wheel frame to hold the wheel and the power transmission system.
    \item To design a motor control circuit for translation and rotational motion for each caster wheels.
    \item To develop algorithms to control the platform and achieve holonomic and omnidirectional motion.

\end{enumerate}


\subsection{Justification of the study}

The design and fabrication of a prototype mobile platform with holonomic and omnidirectional motion is a justified study for several reasons.

Firstly, mobile platforms are an important and growing technology, with a wide range of potential applications in various fields, such as transportation, material handling, logistics, and military operations. However, most mobile platforms currently available are limited to rectilinear or translational motion, meaning they can only move in a straight line or turn about a fixed axis. This limitation restricts their capabilities and limits their potential use in certain scenarios, such as navigating complex or cluttered environments or performing precise movements.

Secondly, omnidirectional and holonomic motion, which allows a mobile platform to move in any direction with equal ease, has the potential to greatly enhance the capabilities and versatility of mobile platforms. This type of motion is typically achieved using specialized wheels or motors, such as mecanum wheels or Mecanum drive systems, which have 3 degrees of freedom and can move in any direction. However, these technologies can be expensive and complex to implement, and may not be suitable for all applications.

Overall, the design and fabrication of a prototype mobile platform with holonomic and omnidirectional motion is a justified study due to the potential benefits and advancements it could bring to the field of mobile platforms, as well as the need for a more affordable and practical solution for achieving this type of motion.

The apparent availability of cheap and accessible technologies that are bridging the gap between holonomic and nonholonomic motion control is the main motivation behind this study.
Holonomic motion is very efficient, with navigation to and from tight spaces being a reality. Furthermore, developing a cheap holonomic mobile platform in the field of mobile
robots, especially in Sub-Saharan Africa aligns with one of the United Nations \ac{UN}
Sustainable Development Goals \ac{SDG} goals, building infrastructure, promoting inclusive and sustainable industrialization, and fostering innovation \cite{noauthor_goal_nodate}. 

\subsection{Expected Outcomes}
\begin{enumerate}
    \item A fully operational mobile platform with a chassis and body
    \item Omnidirectional and holonomic motion control
    \item Unmanned control using a mobile application and a hand motion control device
\end{enumerate}
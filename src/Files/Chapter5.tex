\section{Conclusion}

Holonomic and omnidirectional motion can offer several benefits to mobile robots. Holonomic movements, which are inherently omnidirectional, allow a mobile robot to move in any direction without having to change its orientation. 

One disadvantage of rectilinear motion is that the robot can only move in a straight line, making it difficult to navigate around obstacles or turn corners. In contrast, this mobile platform allows the robot to move in any direction without having to change its orientation, making it more maneuverable and better suited for navigating complex environments.

There are also some potential challenges to incorporating holonomic and omnidirectional motion into mobile robots. One challenge is the additional complexity of the design, as it requires more wheels and may require more advanced control algorithms to coordinate the movement of these wheels. Additionally, the cost of implementing holonomic and omnidirectional motion may be higher due to the increased number of wheels required.

The shortcomings that arise from mobile robots having rectilinear motion similar to conventional automobiles were presented in this report. A proposal was made to incorporate holonomic and omnidirectional motion into mobile robots and finally,a prototype of the holonomic and omnidirectional mobile robot was designed and fabricated. 

\par
In conclusion, the design and fabrication of the prototype mobile platform with holonomic and omnidirectional motion was a successful and promising project. The results of this process matched all the objectives listed in Section \ref{sec:objectives} that guided the project. Two of the three expected outcomes were met as follows:
\begin{enumerate}[i.]
    \item A fully operational mobile platform with a chassis and a body was fabricated
    \item This mobile platform was made to exhibit omnidirectional and holonomic motion
\end{enumerate}

The third expected outcome which involved controlling the platform remotely and unmanned using a mobile application and a hand motion control device fell short due to failure in data transmission. Data in the form of yaw and pitch values was generated using a joystick on the mobile application and used to drive a single motor as in Figure \ref{fig:pwmResult} but transmission troubles curtailed the final objective. This prompted the following recommendations on future research work:
\begin{enumerate}[i.]
    \item Development of novel algorithms for motion planning and control of omnidirectional mobile robots
    \item Evaluation of the control challenges associated with holonomic and omnidirectional motion in mobile platforms
\end{enumerate}

However, there were some challenges that were encountered during the project. One of the main challenges was instability, as the platform was prone to tipping over when moving at high speeds or carrying a heavy payload. This issue can be partially addressed by carefully calculating the dimensions and weight distribution of the platform, but further improvements could be made by adding stability features such as gyroscopes or counterweights.

Despite seemingly achieving all the objectives, the process was full of shortcomings that made the results fail to meet the desired standards. For instance, challenges arose due to old machines, and outdated machining processes. The next step would ideally be having more modern fabrication equipment and tools. 

Overall, the design and fabrication of the prototype mobile platform with holonomic and omnidirectional motion was a successful and promising project, but there are still some challenges that need to be addressed in order to fully realise the potential of this type of platform. Future research could focus on improving the stability and range of the platform, as well as exploring additional applications and potential improvements to the design.

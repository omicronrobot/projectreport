\addcontentsline{toc}{section}{Abstract}
\section*{Abstract}
\label{sec:abstract}

Mobile platforms are a relatively new technology compared to the advancements made in the motor vehicle industry. When producing mobile platforms, inventors and manufacturers have followed the tested blueprint of the motor vehicle industry, which includes four-wheeled rectilinear or translational motion. There has been little progress in developing platforms with omnidirectional and holonomic motion. The mecanum wheel, which has 3 degrees of freedom and can move in any direction, is the best iteration of this technology. There is a need to develop a mobile platform that can rival the mecanum wheel by providing holonomic motion and having a wider range of applications, especially in the industry.
\par
This study explores the design and fabrication of such a mobile platform. The inspiration for this project comes from the use of caster wheels in various applications such as shopping carts, hospital carts, and in the industry for moving heavy payloads. The main objective was to add some form of control to these caster wheels, which would be achieved using \ac{DC} motors and stepper motors. By varying the speed of the \ac{DC} motors and the angle of turn of the stepper motors, the speed and direction of the caster wheels could be controlled. The steering and driving motions of each wheel are mechanically coupled by belts and are actuated synchronously, ensuring that the wheel orientations are always identical, resulting in omnidirectional motion. The aim was to control these parameters remotely using a mobile application and a hand motion control device. The results of this process were conclusive, and a mobile platform with omnidirectional motion was developed. By utilizing the relatively untapped technology of caster wheels, this project makes a significant contribution to the advancements in omnidirectional and holonomic control of mobile platforms.


